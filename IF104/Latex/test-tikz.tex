\documentclass[a4paper]{article}

\usepackage[utf8]{inputenc}
\usepackage[french]{babel}
\usepackage{graphicx}
\graphicspath{{./img/}}
\usepackage{amsmath, amssymb}
\usepackage[left=3cm,right=3cm,top=2cm,bottom=2cm]{geometry}
\usepackage{url}
\usepackage[french]{babel} 

\usepackage{tikz}
\usetikzlibrary{arrows,automata,shapes}

\newcommand{\pgflogo}{\textsc{Pgf}}
\newcommand{\tikzlogo}{Ti\emph{k}Z}
\newcommand{\pgftikz}{\pgflogo\slash\tikzlogo}

\begin{document} 

\begin{center}
  {\LARGE \pgftikz}
\end{center}
 
\section{Le minimum à connaître}

Une ligne
\tikz \draw (0pt, 0pt) -- (20pt, 10pt);
une boucle, 
\tikz \draw (0, 0) -- (1, 0) -- (1, 1) -- cycle;  
avec l'environnement \texttt{tikzpicture} 
\begin{tikzpicture} 
  \tikz \draw[thick,rounded corners=8pt]  
  (0,0) -- (0,2) -- (1,3.25) -- 
  (2,2) -- (2,0) -- (0,2) --  
  (2,2) -- (0,0) -- (2,0);
\end{tikzpicture} 
mais comme il n'y a q'une seule instruction c'est équivalent à 
\tikz \draw[red,thick,rounded corners=8pt] 
(0,0) -- (0,2) -- (1,3.25) --   
(2,2) -- (2,0) -- (0,2) --  
(2,2) -- (0,0) -- (2,0);  
sauf que là c'est rouge.  
 
\section{Exemples plus complexes} 
  
\subsection{Lignes}  
\begin{figure}[!htbp]
  \centering 
  \input{tikz-maze}
  \caption{Un labyrhinte}
\end{figure}

\newpage 
\subsection{Fonctions}
\begin{figure}[!htbp]
  \centering
  \begin{tikzpicture}
  \draw[very thin,color=gray] (-5,-5) grid (5,8);
  \draw[->] (-5.2,0) -- (5.2,0) node[right] {$x$}; 
  \draw[->] (0,-5.2) -- (0,8.2) node[above] {$f(x)$};
  \draw[color=red]    
    plot (\x,\x) node[right] {$f(x)=x$}; 
  \draw[color=blue] 
    plot (\x,{0.05*exp(\x)}) node[right] {$f(x) = \frac{1}{20}\exp(x)$};
\end{tikzpicture}
  \caption{Des fonctions}
\end{figure}


\newpage 
\section{Automate}

\begin{table}[!htbp]
  \centering
  \begin{tabular}{cc}
    \texttt{tikz-automata.tex}&
    \texttt{tikz-automata-bis.tex}\\
    \hline
    %\begin{tikzpicture}[->,>=stealth',shorten >=1pt,auto,node distance=3cm,
%                    semithick]
\begin{tikzpicture}[->,>=stealth, shorten >=1pt,auto,node distance=3cm,
                    semithick]
% -> : une fleche, >=stealth : le style du bout de la flèche,
% shorten>=1pt : distance au bout de la fleche, auto : placement
% automatique, node distance = 3cm : distance antre les noeuds
  \tikzstyle{every state}=[fill=red!50, draw=none, text=black]
% chaque noeud, remplissage a 50% de transparence, pas de bord, le
% texte de couleur noir 

  \node[initial,state] (A)                    {$q_a$};
  \node[state]         (B) [above right of=A] {$q_b$};
  \node[state]         (C) [below right of=B] {$q_c$};

  \path (A) edge              node {0,1,L} (B)
            edge              node {1,1,R} (C)
        (B) edge [loop above] node {1,1,L} (B)
            edge [bend left]  node {0,1,L} (C)
            edge   node {0,1,L} (C);
\end{tikzpicture}&
    \input{tikz-automata-bis} 
  \end{tabular}
  \caption{
    \`A gauche: l'automate donné en exemple, à droite : la modification
    demandée
  }
  \label{tab:tikz-automata}
\end{table}

\section{Arbre}

\begin{table}[!htbp]
  \centering
  \begin{tabular}{cc}
    \texttt{tikz-tree.tex}&
    \texttt{tikz-tree-bis.tex}\\
    \hline
    \input{tikz-tree}&
    \begin{tikzpicture}[level distance=20mm] 
  \tikzstyle{every node}=[fill=red!90,circle,inner sep=1pt] 
  \tikzstyle{level 1}=[sibling distance=30mm, 
  set style={{every node}+=[fill=red!60]}] 
  \tikzstyle{level 2}=[sibling distance=20mm,
  set style={{every node}+=[fill=red!45]}] 
  \tikzstyle{level 3}=[sibling distance=10mm,
  set style={{every node}+=[fill=red!15]}]
  
  \node {\texttt{/}} 
    child {node {\texttt{usr/}}
    } 
    child {node {\texttt{etc/}} 
      child {node {\texttt{X11}} 
        child[fill=none] 
        {edge from parent[draw=none]}
        child {node {\texttt{X}}} 
      }
    };
\end{tikzpicture}
 
  \end{tabular}
  \caption{
    \`A gauche: l'arbre donné en exemple, à droite : la modification
    demandée
  }
  \label{tab:tikz-tree}
\end{table}


\end{document}
