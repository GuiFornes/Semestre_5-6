%%
%% @filename TD1.tex
%% @date mar. 03 nov. 2020 10:36:59 CET
%% @author Guillaume Fornes <guillaume.fornes@enseirb-matmeca.fr>
%%
\documentclass[a4paper, draft]{article}

\usepackage[utf8]{inputenc}
\usepackage[french]{babel} 

% Figures
\usepackage{graphicx}
\graphicspath{{./img/}}

% Math
\usepackage{amsmath, amssymb}
\newtheorem{defi}{Définition}

% Algortihmes
\usepackage[vlined,lined,linesnumbered,boxed,french]{algorithm2e}
\DeclareMathOperator*{\argmin}{argmin}
\DeclareMathOperator{\myfunc}{myfunc}
\DeclareMathOperator*{\sign}{sign}
\DeclareMathOperator*{\imwh}{width}
\DeclareMathOperator*{\imht}{height}

% Extra
\usepackage[left=3cm,right=3cm,top=2cm,bottom=2cm]{geometry}
\usepackage{url}

\begin{document}

\section*{TD1 - Logique propositionelle}

\subsection*{Exercice 1}
$\text{p : a-chats}\Longrightarrow\text{q : a-chien}$\\
\begin{enumerate}
  \item $p\Longrightarrow q \ \ \iff \ \ \lnot q \Longrightarrow \lnot p \ \ \iff \ \ \lnot q \lor p$ donc c'est vrai\\

  \item C'est faux car $\Longrightarrow$ n'est pas commutative\\

  \item a) Non (réciproque de la contraposée)\\
    b) Oui (contraposéee)\\
    c) Non (réciproque)\\

  \item  $\ p\Longrightarrow q \ \iff \lnot q \Longrightarrow \lnot p$ \\
    $\lnot q\Longrightarrow \lnot p $ \\
     $\ q \Longrightarrow p $
\end{enumerate}


\subsection*{Exercice 2}

\begin{enumerate}
  \item $(p\land q) \lor r$, est satisfaisable car :\\

    \begin{tabular}
      {|c c c|c|}
      \hline
      p&q&r&$(p\land q) \lor r$\\
      \hline
      1&1&1&1\\
      1&1&0&1\\
      1&0&1&1\\
      1&0&0&0\\
      0&1&1&1\\
      0&1&0&0\\
      0&0&1&1\\
      0&0&0&0\\
      \hline
    \end{tabular}\\


  \item $p\Longrightarrow (q \Longrightarrow p)$, est valide car :\\

    \begin{tabular}
      {|c c|c|c|}
      \hline
      p&q&$q\Longrightarrow p$&$p\Longrightarrow (q \Longrightarrow p)$\\
      \hline
      1&1&1&1\\
      1&0&1&1\\
      0&1&0&1\\
      0&0&1&1\\
      \hline
    \end{tabular}\\


  \item $(p\land (\lnot q)) \land r$, est satisfaisable car :\\

    \begin{tabular}
      {|c c c|c|}
      \hline
      p&q&r&$(p\land (\lnot q)) \land r$\\
      \hline
      1&1&1&0\\
      1&1&0&0\\
      1&0&1&1\\
      1&0&0&0\\
      0&1&1&0\\
      0&1&0&0\\
      0&0&1&0\\
      0&0&0&0\\
      \hline
    \end{tabular}\\


  \item $(p\lor q) \land (\lnot p \lor r ) \land ( \lnot q \lor p  ) \land ( \lnot q \lor r  ) \land ( \lnot p \lor \lnot r )$, est non satisfaisable car :\\
    \begin{tabular}
      {|c c c|c|c|c|c|c|c|}
      \hline
      p&q&r&$(p\lor q)$&$\lnot p \lor r$&$\lnot q \lor p$&$\lnot q \lor r$&$\lnot p \lor r$&$(p\lor q)\land (\lnot p \lor r)\land(\lnot q \lor p)\land(\lnot q \lor r)\land(\lnot p \lor r)$\\
      \hline
      1&1&1&1&1&1&1&0&0\\
      1&1&0&1&0&1&0&1&0\\
      1&0&1&1&1&1&1&0&0\\
      1&0&0&1&0&1&1&1&0\\
      0&1&1&1&1&0&1&1&0\\
      0&1&0&1&1&0&0&1&0\\
      0&0&1&0&1&1&1&1&0\\
      0&0&0&0&1&1&1&1&0\\
      \hline
    \end{tabular}

\end{enumerate}
\subsection*{Exercice 3}
\begin{enumerate}
  \item $\phi_{A} \ : \ D + G$ \\
    $\phi_{B} \ : \ \overline{ D} = \lnot D$ \\
    $\phi_{C} \ : \ (\phi_{A}\land\phi_{B}) \lor (\lnot\phi_{A}\land\lnot\phi_{B})$\\
    $\text{   }(A+B).\overline{(A.B)}=((D+G)+\overline{D}).\overline{((D+G).\overline{D})}=1.\overline{(D.\overline{D}+G.\overline{D})}=\overline{D}.G$\\

  \item Table de vérité :\\

    \begin{tabular}
          {|c c|c|c|c|}
          \hline
          D&G&A&B&C\\
          \hline
          1&1&1&0&0\\
          1&0&1&0&0\\
          0&1&1&1&1\\
          0&0&0&1&0\\
          \hline

    \end{tabular}\\

  \item a) non\\
    b) non\\
    c) non
    d) oui
\end{enumerate}



\end{document}
