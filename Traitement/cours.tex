\documentclass[a4paper]{report}
\usepackage{amsmath}
\usepackage{amsfonts}
\usepackage{amssymb}
\usepackage[top=2cm,bottom=2cm,left=2cm,right=2cm]{geometry}

\title{Traitement de l'information}
\begin{document}
\begin{titlepage}
  \begin{center}
    ENSEIRB-MATMECA
    \vspace{6cm}


  \line(1,0){400} \\[0.4cm]

    \begin{Huge}
      T{\huge RAITEMENT} D{\huge E L'}I{\huge NFORMATION}

    \end{Huge}
    \vspace{0.4cm}
  \line(1,0){400} \\[0.4cm]
  \vspace{15cm}
  semestre 5 --- 2020-2021
  \end{center}
\end{titlepage}
\tableofcontents
\chapter{Introduction}
\section{Constitution du tableau de donnée}

\begin{tabular}
  {|c|c|c|c|c|c|c|c|c|c|}
  \hline
  Classe & Survivant & Nom & Sexe & Age & Parent présents & Ref billet & Prix billet & Port \\
  \hline
  \hline
  1&1&Allen Elisabeth&F&29&0&null&211&B5&S\\
  \hline
  3&0&Dyker Adolf&N&23&1&null&7&null&C\\
  \hline
  3&1&Dyker Anna&F&22&1&null&null&null&C\\
  \hline
  3&0&Emir Farell&N&null&0&null&null&null&null\\
  \hline
\end{tabular}

Toute les variables doivent être du même type ($\rightarrow$ quels sont les types de var ?) \\
\begin{itemize}
  \item Quid des données manquantes ? ($\rightarrow$ les remplacer ? par la moyenne ?)
    \item Quid des donnée abérrantes ?
  \item Lignes redondantes ? ($\rightarrow$ à supprimer ?)
    \item Colonnes redondantes ?
\end{itemize}
\section{Les variables et les individus}
\subsection{les individus}
Eléments de la population étudiée
\subsection{les variables}
Application definie par
  $$ \omega \text{(population)} \longrightarrow O \text{(espace d'observation) struct de cet espace}$$
  $1^{ere}$ typologie \\


\begin{tabular}
  {|c|c|c|l|}
  \hline
  card O \/ struct de O & O ensemble continu de $\mathbb{R}$ & O fini ou denombrable & . \\
  \hline
  sans struct =, $\ne$ & / & C8P lieu de residence & nominale \\
  \hline
  struct ordi. $\leq$  & Age imperative & rang note & ordinal \\
  \hline
  corps ordonnée & salaire & / & mesurable \\
  \hline
  . & quantifiable & qualifiable & . \\
  \hline
\end{tabular}
\vspace{1cm}

$2^{eme}$typologie : $\rightarrow $variable d'incidence\\
\hspace{4cm}$\rightarrow$ variable relationelle

\begin{itemize}
  \item variable d'incidence\\
    $\rightarrow$ attribut descriptif ,reponse oui = 1 OU non = 0\\
    $\rightarrow$ var. numérique  espace d'observation = $\mathbb{R}$
  \item var. relationnelles \\
    $\rightarrow$ var rang $\rightarrow$ échelle de notes suffisamment fine pour que 2 individu n'aient pas (pas de suite c'est effacé)\\
    $\rightarrow$ var. présentant des modalités non ordonnées (bac, csp, ...)\\
    $\rightarrow$ var. mesures sur $\mathbb{R} \times \mathbb{R}$
\end{itemize}

\section{Les types de tableaux}
CN   Tableaux doivent être \underline{homogènes} 1 seul types de variable
\subsection{Tableau de données quantitatives}
(pourcentage de minerai par sondage / profondeur)
$$x^{j\leftarrow colonne}_{i\leftarrow ligne}$$
\underline{traitement}
\subsection{tableau de contingence}
croisement de 2 var qualitatives \\
\underline{Traitement} : Analyse fonctionelle des correspondances
\subsection{tableaux binaires (logique d'incidence)}
tableaux d'attributs descriptif de 1,0 \\
\underline{Traitement} : classifications hiérarchiques\\
\subsection{Tableaux de préferences}
Notes données à des marques de parfum\\
\underline{Traitement} : Analyse fonctionelle des correspondances multiples
\subsection{Tableaux de modalités}

lisez vous tel journal ? \\
\begin{itemize}
  \item 5 : tout le temps
  \item 4 : regulièrement
  \item 3 : parfois
  \item 2 : rarement
  \item 1 : jamais
\end{itemize}

\subsection{tableaux de proximité}
mesure sur $\mathbb{R}\times\mathbb{R}$

\begin{tabular}
  {c|c|c|c}
  &r1&r2&r3\\
  \hline
  r1&&&\\
  \hline
  r2&&&\\
  \hline
  r3&&&\\

\end{tabular}

\section{Changement de variables}
$\rightarrow$ pour rendre le tableau homogène \hspace{4cm}
\begin{tabular}
  {c|c|c|}
  &Sexe&couleur yeux\\
  \hline
  pere&M&marron\\
  \hline
  mere&F&bleu\\
  \hline
  enfant&M&vert\\
\end{tabular}\\

codage disjonctif complet\\

\begin{tabular}
  {c|c|c|c|c|c|}
  &sexe M&sexe F & yeux marr & yeux bleu & yeux vert \\
  \hline
  pere&1&0&1&0&0\\
  mere&0&1&0&1&0\\
  enfant&1&0&0&0&1\\
  \hline
\end{tabular}\\

$\longrightarrow$ \underline{Tableau de \emph{Burt}}\\

$^{t}D \times D$\\


\begin{tabular}
  {c|c|c|c|c|c|}
  &sexe M&sexe F&Y m&Y b&Y v\\
  \hline
  sexe M&2&0&1&0&1\\
  \hline
  sexe F&0&1&0&1&0\\
  \hline
  Y m&1&0&1&0&0\\
  \hline
  Y b&0&1&0&1&0\\
  \hline
  Y v&1&0&0&0&1\\
  \hline
\end{tabular}\\

Tableaux d'effectifs $\longrightarrow$ Analyse fonctionnelle des correspondances\\

autre exemple : Tableau de Burt du Titanic\\

\begin{tabular}
  {c|c|c|c|c|c|c|c|c|c}
  &classe 1&classe 2&classe 3&enf&adul&F&M&surv 0&surv 1\\
  \hline
  classe 1&325&0&0&6&319&147&180&122&203\\
  \hline
  classe 2&0&285&0&24&261&106&179&167&118\\
  \hline
  classe 3&0&0&706&79&627&null&null&null&null\\
  etc&$\vdots$&$\vdots$&$\vdots$&$\vdots$&$\vdots$&$\vdots$&$\vdots$&$\vdots$&$\vdots$\\
\end{tabular}

\section{Elements descriptifs d'un tableau de données}
n : individus/lignes\\
p : variable/colonnes\\

individu i = $x_{i}= \begin{pmatrix}
  x_{i}^{1}&
  \hdots&
  x_{i}^{p}
\end{pmatrix}
\in \mathbb{R}^{p}$
la variable j : $x_{j}= \begin{pmatrix}
  x_{i}^{1} \\
  \vdots \\
  x_{i}^{p}
\end{pmatrix}
\in \mathbb{R}^{n}$\\
On va étudier le nuage $N_{I}=\{x_{i} , i=1,\hdots,n, \text{poids} p_{i}=\dfrac{1}{n}\}$\\
 $N_{J}=\{x_{j} , j=1,\hdots,p , \text{poids} p_{j}=\dfrac{1}{p}\}$\\

 \underline{Soit le nuage des variables}

 A chaque variable, on peut associer:
 sa myenne
 $$\underline{x}^{j}=\sum^{n}_{i=1}p_{i}x_{i}^{j}$$
 sa variance
 $Var\underline{x}^{j}=\sum^{n}_{i=1}p_{i}(x_{i}^{j}-\overline{x}^{j})^{2}$

\vspace{2cm}
 TROU
\vspace{2cm}


 Pour le nuage $N_{I}$on peut calc le centre de garvité :\\
 $$\underline{\overline{x}}=\sum_{i=1}^{n}p_{i}\underline{x_{i}}$$
 $$\underline{\overline{x}}=\begin{pmatrix}
   \overline{x}^{1}\\
   \vdots\\
   \overline{x}^{p}
 \end{pmatrix}
 $$
 Sur le nuage $N_{j}$ on peut calculer le centre de gravité $$\underline{\overline{x}}=\begin{pmatrix}
   \overline{\underline{x}}_{1}\\
   \vdots\\
   \underline{\overline{x}}_{n}
 \end{pmatrix}$$
 \underline{L'inertie du nuage}

 des individus  $I_{N_{I}}=\sum^{n}_{i=1}p_{i}d^{2}(x_{i},\underline{\overline{x}})$
 $\longrightarrow$ \underline{distance}

 \section{Choix d'une mesure de ressemblance}
 \underline{définition 1} : indice de similarité
 $$ s \Omega\time\Omega\longrightarrow\mathbb{R}^{+}$$
 s est un indice de similarité si :
 \begin{enumerate}
   \item s est symetrique, $s(x, y)=s(y, x)$
   \item $s(x, x)=s(y, y) \geq s(x, y)$
 \end{enumerate}
\underline{définition 2} : indice de dissimilarité
$$ s \Omega\time\Omega\longrightarrow\mathbb{R}^{+}$$
s est un indice de dissimilarité si :
\begin{enumerate}
  \item symetrique
    \item s(x, x)=0

\end{enumerate}
\underline{definition 3} : une distance est un indice de dissimilarité\\
qui vérifie en plus
 \begin{enumerate}
   \item $s(x, y)=0 \leftrightarrow x=y$
   \item $s(x, z)\leq s(x,y)+s(y,z)$
 \end{enumerate}
 \begin{itemize}
     \item distance euclidienne (A en composantes principales)
   \item distance dite du $\chi^{2}$ (Analyse factorielle des correspondances)

     $$d^{2}(\underline{x}_{i}, \underline{x}_{i'})=\sum_{j=1}^{p}\dfrac{1}{x^{j}_{i}}(\dfrac{x^{j}_{i}{x_{i}}}-\dfrac{x_{i'}^{j}}{x_{i'}})^{2}$$
   \item Tableau d'incidence

     TROU

\end{itemize}
\underline{Distance entre groupes}

(indice d'aprépation)

distance du lien max

%$\delta(A, B)=max_{x\inA \& y\inB}(d(x, y))$\\
distance du lien min \hspace{5cm} distance fréquemment utilisé\\
distance des centres de gravité \hspace{4cm} $\delta(A, B)=I$
$\delta(A, B)=d(\underline{\overline{x}}, \underline{\overline{y}})$

trou mais fin de chap

\chapter{Analyse en composantes principales : \emph{ACP}}

Traite les tableaux de données quantitativement positives

\section{Principe de la méthode}


Le pb qui se pose dans la mesure où 1 partie des variables sont liées, c'est de passer d'un tableau X de dimension $n \times p$ à un tableau Y de dim $n \times q$ $q < p$ en réduisant le nbr de variable descriptive tout en perdant le moins possible d'info\\
la methode utilisé en ACP pour passer de p var à q var q<p consiste à projeter le nuage des individus sur un sous espace W de dim q en deformant le moins possible le nuage lors de sa projection\\
Les composantes principales sont les nouvelles variables. elles vont s'interpreter comme des'' 'synthèses' des variables initiales $\underline{x^{j}}$.\\

\vspace{1cm}
TROU flemme 3 lignes + fin du cours
\vspace{1cm}

Le principe de l'ACP, c'est de réduire le nbr de variables décrivant les individus en perdant le moins possible d'information

\section{Formalisation du problème}


Rechercher l'espace de projection \emph{W} (a : origine, $u_{1},\hdots, u_{q}$ : système générateur de W)
de telle façon que la perte d'info fut minimale

$$\forall i \underline{\widehat{x_{i}}}=\underline{a}+\sum^{a}_{k=1}y^{k}_{i}\underline{u_{k}}$$

$\underline{\widehat{x_{i}}}$ est la projection orthogonale de $\underline{x_{i}} sur \emph{W}$

On chercher le referentiel \emph{W} tq $\sum _{W}=\sum^{n}_{i=1}p_{i}d^{2}(\underline{x_{i}}, \underline{\widehat{x_{i}}})$

\section{Resoplution du problème}

\subsection{Choix de a}

\subsection{choix de $[\underline{u_{1}}, \hdots, \underline{u_{q}}]$}




\end{document}
