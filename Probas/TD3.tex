\documentclass[a4paper]{report}
\usepackage{amsmath}
\usepackage{amsfonts}
\usepackage[top=2cm, bottom=2cm, left=2cm, right=2cm]{geometry}
\newcommand{\exo}[1]{\subsection*{exercice #1}}

\begin{document}
\section*{TD3}
\subsection*{exercice 8.  Inégalité de Markov} 
\underline{def} moment absolu d'ordre k $$u_k = E(|X|^{k}$$
Montrons que $$P(|X|\geq\epsilon)\leq\dfrac{1}{\epsilon²}E(|X|^{\alpha})$$

$$ P(|X|\leq\epsilon)=\int_{|X|>=\epsilon}f(x)dx=\frac{1}{\epsilon^{\alpha}}\int_{|X|>=\epsilon}\epsilon^{\alpha}f(x)dx\leq=\frac{1}{\epsilon^{\alpha}}\int_{|X|\geq\epsilon}|\epsilon|^{\alpha}f(x)dx==\frac{1}{\epsilon^{\alpha}}E(|X|^{\alpha}) $$


\subsection*{exercice 10. Distribution triangulaire}
si X et Y 2 v.a. indép. de densité f et g alors $Z=X+Y$ aura pour densit le produit de convulotion
$$f*g(z)=\int_{D}f(x)g(z-x)dx=\int_Dg(y)f(z-y)dy$$
$$f_X(x)=\mathbb{1}_{[0;1]}(x) \text{ et } f_Y(y)=\mathbb{1}_{[0,1]}(y)$$
$$\text{donc  } g_z(z)=\int_{-\inf}^{\inf}\mathbb{1}_{[0,1]}(x)\mathbb{1}_{[0,1]}(z-x)dx$$
$$\text{Pour que } g(z)\ne0 \text{ il faut }x\in[0;1]\text{ et } (z-x)\in[0;1]$$

\textbf{...}

\exo{12.}






\end{document}
